\documentclass[a4paper, 12pt]{article}
\usepackage[T2A]{fontenc}
\usepackage[utf8]{inputenc}
\usepackage[english,russian]{babel}
\usepackage{amsmath, amsfonts, amssymb, amsthm, mathtools, misccorr, indentfirst, multirow}
\usepackage{wrapfig}
\usepackage{graphicx}
\usepackage{subfig}
\usepackage{adjustbox}

\title{Лабораторная работа 4.2.1\\Кольца Ньютона}
\author{Нехаев Александр\\654 группа}
\date{\today}

\begin{document}
	\maketitle
	\pagenumbering{gobble}
	\newpage
	\tableofcontents
	\pagenumbering{arabic}
	\newpage
	\section{Введение}
	\paragraph{Цель работы:} ознакомление с явлением интерференции в тонких пленках (полосы равной толщины) на примере колец Ньютона и с методикой интерференционных измерений кривизны стеклянной поверхности.
	\paragraph{В работе используются:} измерительный микроскоп с опак-иллюми-натором; плосковыпуклая линза; пластинка из черного стекла; ртутная лампа ПРК-4; щель; линзы; призма прямого зрения; объектная шкала.\par
	В нашей установке кольца Ньютона образуются при интерференции световых волн, отраженных от границ тонкой воздушной прослойки, заключенной между выпуклой поверхностью линзы и плоской стеклянной пластинкой (рис. \ref{calc_scheme}). Наблюдение ведется в отраженном свете.\par
	Рассчитаем размер колец Ньютона. Пусть сверху на линзу падает монохроматический параллельный пучок лучей. При вычислении разности хода можно пренебречь небольшими наклонами лучей, проходящих в тонком воздушном зазоре. Геометрическая разность хода между интерферирующими лучами равна, очевидно, $2d$, где $d$ — толщина воздушного зазора в данном месте.\par
	Выразим зависимость $d$ от расстояния $r$ до радиуса, проходящего через точку соприкосновения линзы и пластинки. Из рис. \ref{calc_scheme}. имеем
	\begin{equation*}
		r^2=R^2-\left(R-d\right)^2=2Rd-d^2,
	\end{equation*}
	где R — радиус кривизны выпуклой поверхности линзы. Принимая во внимание, что $2R\gg d$, получим
	\begin{equation}
		d=\frac{r^2}{2R}.
	\end{equation}
	\par
	При вычислении полной разности хода нужно учесть изменение фазы световой волны при отражении от границы стекло-воздух и воздух-стекло. Как известно, для светового (электрического) вектора отражение от оптически более плотной среды происходит с изменением фазы на $\pi$. Свет, отраженный от границы стекло—воздух, по сравнению со светом, отраженным от границы воздух—стекло, приобретает, таким образом, дополнительный фазовый сдвиг на $\pi$, что соответствует разности хода $\lambda/2$. Полная разность хода $\Delta$ равна
	\begin{equation}
		\Delta=2d+\frac{\lambda}{2}=\frac{r^2}{R}+\frac{\lambda}{2}.
	\end{equation}
	Линии постоянной разности хода представляют собой концентрические кольца с центром в точке соприкосновения. При заданном значении длины волны $\lambda$ разность хода $\Delta$ определяется толщиной воздушного зазора; интерференционные полосы являются, таким образом, линиями равной толщины.\par
\end{document}