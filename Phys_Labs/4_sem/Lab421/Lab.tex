\documentclass[a4paper, 12pt]{article}
\usepackage[T2A]{fontenc}
\usepackage[utf8]{inputenc}
\usepackage[english,russian]{babel}
\usepackage{amsmath, amsfonts, amssymb, amsthm, mathtools, misccorr, indentfirst, multirow}
\usepackage{wrapfig}
\usepackage{graphicx}
\usepackage{subfig}
\usepackage{adjustbox}

\title{Лабораторная работа 4.2.1\\Кольца Ньютона}
\author{Нехаев Александр\\654 группа}
\date{\today}

\begin{document}
	\maketitle
	\pagenumbering{gobble}
	\newpage
	\tableofcontents
	\pagenumbering{arabic}
	\newpage
	\section{Введение}
	\paragraph{Цель работы:} ознакомление с явлением интерференции в тонких пленках (полосы равной толщины) на примере колец Ньютона и с методикой интерференционных измерений кривизны стеклянной поверхности.
	\paragraph{В работе используются:} измерительный микроскоп с опак-иллюми-натором; плосковыпуклая линза; пластинка из черного стекла; ртутная лампа ПРК-4; щель; линзы; призма прямого зрения; объектная шкала.\par
	В нашей установке кольца Ньютона образуются при интерференции световых волн, отраженных от границ тонкой воздушной прослойки, заключенной между выпуклой поверхностью линзы и плоской стеклянной пластинкой (рис. \ref{calc_scheme}). Наблюдение ведется в отраженном свете.\par
	
\end{document}