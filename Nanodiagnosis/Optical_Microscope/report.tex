\documentclass[a4paper, 12pt]{article}
\usepackage[T2A]{fontenc}
\usepackage[utf8]{inputenc}
\usepackage[english,russian]{babel}
\usepackage{amsmath, amsfonts, amssymb, amsthm, mathtools, misccorr, indentfirst, multirow}
\usepackage{wrapfig}
\usepackage{graphicx}
\usepackage{subfig}
\usepackage{enumitem}
\usepackage{adjustbox}
\usepackage{pgfplots}
\usepackage{caption}
\usepackage{pgf}
\usepackage{tikz}
\usetikzlibrary{arrows,automata}
\usetikzlibrary{positioning}

\usepackage{geometry}
\geometry{top=20mm}
\geometry{bottom=20mm}
\geometry{left=20mm}
\geometry{right=20mm}
\newcommand{\angstrom}{\textup{\AA}}

\begin{document}
	\begin{titlepage}
		\begin{center}
		МИНИСТЕРСТВО ОБРАЗОВАНИЯ И НАУКИ РОССИЙСКОЙ ФЕДЕРАЦИИ\\
		\footnotesize{Московский физико-технический институт}\\
		\footnotesize{(государственный университет)}\\
		\vfill
		{\LARGE
		\textbf{Оптическая микроскопия}\\
		}
		\vspace{1cm}
		Лабораторная работа по курсу\\
		нанодиагностике
		\vfill
		\begin{flushright}
			Выполнили: студенты 652 группы.\\
            Нехаев А.С.\\
		\end{flushright}
		\vfill
		г. Долгопрудный\\
		\the\year\:год
		\end{center}
	\end{titlepage}
	\newpage
	\pagenumbering{arabic}
	\tableofcontents
    \newpage
	\section{Введение}
	\subsection{Цель работы}
	\begin{enumerate}
		\item Ознакомление с устройством и принципом работы оптического микроскопа BX51M;
		\item Определение максимального разрешения микроскопа в режимах светлого и темного поля;
		\item Определение периода четырех структур в режимах светлого поля и дифференциально-интерференционного контраста;
		\item Определение степени загрязнения образца в режим темного поля.
	\end{enumerate}
	\subsection{Теоретическая часть}
	\subsubsection{Схема микроскопа}
	\subsubsection{Методика наблюдения в отраженном свете по методу светового/темного поля}
	\tikzstyle{block} = [draw, fill=blue!20, rectangle, minimum height=2em, minimum width=6em]
	\tikzstyle{sum} = [draw, fill=blue!20, circle, node distance=1cm]
	\tikzstyle{input} = [coordinate]
	\tikzstyle{output} = [coordinate]
	\tikzstyle{pinstyle} = [pin edge={to-,thin,black}]

	\begin{tikzpicture}[auto, node distance=2cm,>=latex']
		\node [block, name=block1] (block1) {Выберите наблюдение по методу светлого (BF) или темного (DF) поля.};
		\node [block, below of=block1] (block2) {Установите сетевой выключатель в положение <<I>> (Выключено)};
		\draw [->] (block1) -- (block2); 
	\end{tikzpicture}

\end{document}